% Chapter 2

\chapter{古诗生成与大模型}

本章主要介绍古诗生成任务的基本要求、现有技术方案以及古诗质量的度量方法,分析现有研究的不足之处,并简要介绍DeepSeek-R1与VL2两个大模型,为后续的系统设计方案作铺垫。

\section{古诗生成任务概述}
……
\section{现有方案}
……
\section{古诗质量评估}
\subsection{BLEU}
……
\subsection{ROUGE}
……
\subsection{Distinct}
……
\subsection{Similarity}
……
\subsection{人工评估}

由于古诗体裁的高度的艺术性,过去的研究中往往会邀请人类评审来评估古诗的质量。评估往往会基于单独设计的分析角度进行,如“流畅性”、“艺术性”、“连贯性”等等,依赖于分析维度的先验设计。此外,人类评审的结果往往依赖于评审员自身的文化素养、个人品味与喜好,结果难有一致性。此外,招募具有高文学素养的人类评审员也是一个难题。

由此,本文认为可利用大模型的语言能力行使人类评审员的功能,设计一套严格的质量评估体系,作为提示词指导大模型,便可得到具有高解释性、高一致性的古诗质量评估结果。

\section{DeepSeek大模型}
\subsection{DeepSeek-R1}
……
\subsection{DeepSeek-VL2}
……
