% !TeX root = ..\main.tex
% Chapter 4

\chapter{实验及结果分析}

为验证本系统在古诗生成、评价和优化方面的有效性,本文设计开展了相关实验,利用已有古诗数据集来检验系统性能。此外,结合自动度量方法对比了ERNIE-4.0和DeepSeek-R1两种模型产出的结果。

\section{基于白话文的古诗生成实验}

参考相关工作中评估模型表现的方法,基于相同的输入,将评估模型的输出古诗与原古诗进行对比,利用BLEU和ROUGE计算指标分数。在这一思路下,原古诗的质量应当足够好,以确保指标分数能够反映被评估输出的质量;且模型的输入要与原古诗的内容十分相关,以确保指标分数能够反映模型的生成能力。

为此,本文选择《唐诗三百首》中的名篇作为古诗原文,选择相应的白话文翻译作为文本输入。《唐诗三百首》是清代蘅塘退士孙洙编选的唐诗选集,问世不久便闻名遐迩,成为唐诗入门读物的首选,其中收录的唐诗均为名篇佳作。原书中除了律诗、绝句外,还收录有乐府、古体诗等形式多变的体裁,故不便于评估,本文仅选择律诗和绝句两种体裁的古诗进行测试,如表~\ref{tab:test_generation_data}。此外,测试集选择的白话文翻译源自古诗文网\footnote{古诗文网: \url{https://www.gushiwen.cn/gushi/tangshi.aspx/}}。


\begin{table}[ht]
    \centering
    \caption{《唐诗三百首》测试数据集}
    \label{tab:test_generation_data}
    \begin{tabular}{|c|c|}
      \hline
      \bf{体裁}& \bf{数量} \\
      \hline
      七言律诗& 51\\
      \hline
      七言绝句& 50\\
      \hline
      五言律诗& 80\\
      \hline
      五言绝句& 29\\
      \hline
      合计& 210\\
      \hline
    \end{tabular}
  \end{table}

测试发现,ERNIE-4.0和DeepSeek-R1两种模型的输出均与参考古诗有极高的覆盖度,因而在BLEU和ROUGE指标上均有很高的得分,明显区别于以往工作中的结果(如\verb|BLEU-1|=0.168, \verb|BLEU-2|=0.002\cite{chenPolishingModelMachineGenerated2024}),属于异常情况。作为大模型,两种模型的训练数据均包含了大量的古诗数据,尤其是对《唐诗三百首》这样的名篇,因而本实验并不能做到训练集与测试集的独立,并不具备验证效果。
测试统计结果见表~\ref{tab:test_generation_tssbs_dsr1}和表~\ref{tab:test_generation_tssbs_ernie4}。


\begin{table}[ht]
  \centering
  \caption{白话文古诗生成实验结果(DeepSeek-R1)}
  \label{tab:test_generation_tssbs_dsr1}
  \begin{tabular}{cccccc}
      \toprule
      & \multicolumn{2}{c}{BLEU} & \multicolumn{3}{c}{ROUGE} \\
      \cmidrule(r){2-3} \cmidrule(l){4-6}
      & BL-1& BL-2& R-1& R-2& R-L\\
      \midrule
      七言律诗& 0.583599&\bf{0.432841}&\bf{0.557353}&\bf{0.349461}&0.527267\\
      七言绝句&0.559833&0.415325&0.562665&0.344502&0.540417\\
      五言律诗&\bf{0.597726}&0.432494&0.549401&0.337500&\bf{0.545573}\\
      五言绝句&0.523605&0.332445&0.495130&0.245845&0.471983\\
      \cmidrule{2-6} %添加2-4列的中线
      平均&0.575037&0.414674&0.546996&0.329415&0.529737\\
      \bottomrule
  \end{tabular}
\end{table}

\begin{table}[ht]
  \centering
  \caption{白话文古诗生成实验结果(ERNIE-4.0)}
  \label{tab:test_generation_tssbs_ernie4}
  \begin{tabular}{cccccc}
      \toprule
      & \multicolumn{2}{c}{BLEU} & \multicolumn{3}{c}{ROUGE} \\
      \cmidrule(r){2-3} \cmidrule(l){4-6}
      & BL-1& BL-2& R-1& R-2& R-L\\
      \midrule
      七言律诗&0.765951&0.671162&0.770484&0.609392&0.743464\\
      七言绝句&0.798136&0.702556&0.808346&0.649742&0.764375\\
      五言律诗&0.691980&0.533006&0.670557&0.445450&0.646720\\
      五言绝句&\bf{0.839312}&\bf{0.764698}&\bf{0.834125}&\bf{0.726052}&\bf{0.812500}\\
      \cmidrule{2-6} %添加2-4列的中线
      平均&0.767420&0.658670&0.765072&0.596372&0.734993\\
      \bottomrule
  \end{tabular}
\end{table}

\section{评分功能实验}

为了检验系统评分功能的有效性,本文尝试收集具有质量差异的古诗分组,通过先验的质量分层来验证系统评分的合理性。此外,也将使用BLEU等自动度量方法,作为实验的参考指标。

选择第六届“诗词中国”传统诗词创作大赛的公开获奖作品为测试集,测试系统评分功能的有效性和可信度。该大赛的评审专家均为古诗词领域的专家,且其评分标准公开透明,因而可以作为测试系统评分功能的参考。

测试发现,不同奖项组之间的评分存在差异但并不显著,且由于各奖项样本数量分布不均(一等奖2首,二等奖8首,三等奖18首,优秀奖91首),结论难有普适性。而往届获奖作品并不公开,因而无法获取更多数据。因此,将一、二、三等奖的作品进行合并,略微放松比较粒度,发现其与优秀奖的评分存在差异,且比较结果与自动度量指标的结果一致(如图~\ref{tab:test_scoring_pprized_dsr1})

TODO 数据:获奖作品的评分实验

\begin{table}[ht]
  \centering
  \caption{获奖作品的评分实验(DeepSeek-R1)}
  \label{tab:test_scoring_pprized_dsr1}
  \begin{tabularx}{\textwidth}{cccccccccccc}
      \toprule
      &  & \multicolumn{2}{c}{BLEU} & \multicolumn{3}{c}{ROUGE} & \multicolumn{2}{c}{Similarity} & \multicolumn{2}{c}{Distinct}& \\
      \cmidrule(r){3-4} \cmidrule(l){5-7} \cmidrule(l){8-9} \cmidrule(l){10-11}\\
      &	总分 & BL-1& BL-2& R-1& R-2& R-L& S-Intra& S-Inter& D-1& D-2& 样本数\\
      \midrule
      一等奖&	0.820000 	&	0.968750 	&	0.678427 	&	0.096657 	&	0.001543 	&	0.134217 	&	0.675846 	&	0.722989 	&	0.906250 	&	1.000000 	&	2	\\
      二等奖&	0.770000 	&	0.953125 	&	0.805106 	&	0.108463 	&	0.004658 	&	0.142648 	&	0.660678 	&	0.697867 	&	0.890625 	&	1.000000 	&	8	\\
      三等奖&	0.825000 	&	0.938999 	&	0.788950 	&	0.119148 	&	0.004140 	&	0.149292 	&	0.658698 	&	0.688951 	&	0.879656 	&	0.998677 	&	18	\\
      优秀奖&	0.796290 	&	0.934612 	&	0.749390 	&	0.107799 	&	0.003346 	&	0.143840 	&	0.670973 	&	0.696803 	&	0.879247 	&	0.999832 	&	91	\\
      
      \cmidrule{2-11} %添加2-4列的中线
      一二&	0.784286 	&	0.957589 	&	0.768912 	&	0.105090 	&	0.003768 	&	0.140239 	&	0.665012 	&	0.705045 	&	0.895089 	&	1.000000 	&	10	\\
      一二三&	0.810000 	&	0.945848 	&	0.781568 	&	0.113968 	&	0.004003 	&	0.145957 	&	0.661024 	&	0.694880 	&	0.885342 	&	0.999165 	&	28	\\
      
      \cmidrule{2-11} %添加2-4列的中线
      平均&	0.799506 	&	0.937247 	&	0.756938 	&	0.109246 	&	0.003500 	&	0.144337 	&	0.668639 	&	0.696352 	&	0.880676 	&	0.999675 	&	119	\\

      \bottomrule
  \end{tabularx}
\end{table}


为进一步验证,本文扩大测试古诗范围,从古诗文网挑选“打油诗”,结合《唐诗三百首》的部分古诗(五言律诗和七言律诗),与原先的获奖作品一同作为测试集。测试发现,系统的评分比较结果与预设的组别质量差异一致,也与自动度量指标的比较结果一致,验证了评分功能的有效性。

TODO 数据:唐诗、打油诗的评分实验

\section{文图结合的古诗生成实验}

为验证图片模态的必要性,需要设置文图输入的消融对比实验。之前的工作表明,面对不同的输入模态组合,ERNIE-4.0的输出古诗质量会有显著的差异。本文尝试对DeepSeek-R1进行相同的实验,即固定文本和图像输入,分别测试仅文字、仅图像、文字与图像三种模态输入,对比生产的古诗质量。古诗的生成质量依据系统自身的评分功能。

例如,对图~\ref{fig:example_input}中的输入,分别单独输入文本和图像,再利用系统自身的评分以及自动度量方法,比较生成古诗的质量差异。

TODO 数据:模态组合的评分实验
\begin{table}[ht]
  \centering
  \caption{文图古诗生成实验结果}
  \label{tab:test_generation_modal}
  \begin{tabular}{ccccccc}
      \toprule
      & 格律规范&	意象意境&	主题思想&	语言锤炼&	创新维度&	总分\\
      \midrule
      文& 0.92& 0.93& 0.90& 0.93& 0.90& 0.92 \\
      图& 0.92& 0.90& 0.90& 0.87& 0.80& 0.89 \\
      文图& 0.84& 0.90& 0.85& 0.87& 0.80& 0.86\\
      \bottomrule
  \end{tabular}
\end{table}

实验发现,文图模态输入对古诗生长质量的影响并不显著,而这部分差异也属于正常波动。换言之,文图模态输入提供的信息未展现出提高古诗生成质量的作用。

图像输入的作用体现在其包含的视觉场景信息,能够作为用户文本输入的补充,帮助用户表达隐晦的场景情感,以提高用户的体验。但就古诗生成的质量而言,DeepSeek-R1的能力足以生成高分数的古诗,这一点与用户的需求是不同的。