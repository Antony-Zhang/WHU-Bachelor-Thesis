% Chapter 4

\chapter{实验及结果分析}

为验证本系统在古诗生成、评价和优化方面的有效性,本文设计开展了相关实验,利用已有古诗数据集来检验系统性能。此外,结合自动度量方法对比了ERNIE-4.0和DeepSeek-R1两种模型产出的结果。本章首先……

\section{基于白话文的古诗生成实验}

……实验思路和目的……

选择《唐诗三百首》的古诗原文作为参考,选择相应的白话文翻译作为文本输入。将系统生成的古诗与古诗原文作对比,利用BLEU和ROUGE两种方法进行评估。

白话文翻译来自古诗文网\footnote{古诗文网: \url{https://www.gushiwen.cn/gushi/tangshi.aspx/}}。


测试发现,ERNIE-4.0和DeepSeek-R1两种模型的输出均与参考古诗有极高的覆盖度,因而在BLEU和ROUGE指标上均有很高的得分,明显区别于以往工作中的结果(如\verb|BLEU-1|=0.153, \verb|BLEU-2|=0.002)。

……三线表-实验结果……

作为大模型,两种模型的训练数据均包含了大量的古诗数据,尤其是对《唐诗三百首》这样的名篇,因而本实验并不能做到训练集与测试集的独立,因而不具备验证效果。


\section{评分功能实验}

选择第六届“诗词中国”传统诗词创作大赛的公开获奖作品为测试集,测试系统评分功能的有效性和可信度。该大赛的评审专家均为古诗词领域的专家,且其评分标准公开透明,因而可以作为测试系统评分功能的参考。

经测试发现,不同奖项组之间的评分存在差异但并不显著,且由于各奖项样本数量分布不均(一等奖2首,二等奖8首,三等奖18首,优秀奖91首),结论难有普适性。 而往届获奖作品并不公开,因而无法继续这一方向的测试。

……结果……


为此,本文扩大测试古诗范围,从古诗文网挑选“打油诗”,结合《唐诗三百首》的部分古诗,与原先的获奖作品一同作为测试集。

……结果……

\section{文图结合的古诗生成实验}

为验证