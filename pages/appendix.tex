% 附录

\appendix
% \appendixtitleformat  % 切换附录标题格式
\ctexset{
  chapter = {
    name = {附录},         % 设置章节名前缀为“附录”
    number = \Alph{chapter} % 使用 A, B, C 编号
  }
}

% \chapter{提示词}

% 此处列举本文设计的评分体系与提示词。

% \section{古诗评分体系} \label{sec:appendix_poem_scoring_system}

% 下面是本文设计的古文评分体系,在实际使用时将作为连续的文本嵌入“古诗生成”提示词(见图~\ref{fig:prompt_poem_generation})。

% \section{提示词设计} \label{sec:appendix_prompt}
% 此处展示本文设计的提示词。%,内容较长或需要变量插入者也展示了主要结构。%,包括图像分析、古诗生成、古诗优化和古诗评分。








% \clearpage

% \section{古诗评分示例} \label{sec:appendix_poem_analysis_example}
% 此处展示在“古诗评分”提示词中使用到的两个输入输出示例。



% \chapter{数据} \label{chap:appendix_data}
% 除前文已经提及的外部数据集外,此处列举了本项目使用的其他数据集,即在“评分功能实验”中使用到的“打油诗”。


\chapter{成果}

\begin{enumerate}
    \item Yang L, Zhang Z, Niu K, et al. Large Model Based Crossmodal Chinese Poetry Creation[A]. 2024 IEEE Smart World Congress (SWC)[C], Nadi, Fiji: IEEE, 2024 : 27 - 34.
\end{enumerate}


% \section{第一个测试}
% 测试公式编号
% \begin{equation}
%   1+1=2.
% \end{equation}

% 表格编号测试

% \begin{table}[h]
%   \centering
%   \caption{测试表格}
%   \begin{tabular}{*{20}c}
%     \hline
%     11 & 13  & 13  & 13  & 13 \\
%     12 & 14  & 13  & 13  & 13 \\
%     \hline
%   \end{tabular}
% \end{table}