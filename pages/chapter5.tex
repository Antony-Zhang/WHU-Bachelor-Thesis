% Chapter 5

\chapter{结语}
古诗作为中华优秀传统文化的重要组成部分,凝聚了丰富的审美表达、文化意象与哲理思想,既是语言的艺术,也是精神的载体。近年来,随着自然语言处理技术的飞跃发展,越来越多的研究开始尝试借助人工智能模型复现古典文学的创作过程,实现对古诗的自动生成。大语言模型的语义理解与生成能力,多模态模型对图像与语言的协同建模能力,为这一目标提供了全新的技术基础。然而,古诗独特的格律结构、典故运用与意象传达,使其生成任务不仅对模型能力提出更高要求,也对系统的设计方法和使用体验构成挑战。尤其是如何让用户清晰表达创作意图,并从多角度理解生成结果的文化内涵,是目前大多数古诗生成系统仍待解决的问题。


% TODO 研究内容的结构需要明确
围绕上述背景,本文以“基于大模型的跨模态古诗创作”为核心任务,围绕“文图融合输入”、“结构化提示控制”、“自动化质量评估”与“可解释性优化”等四个方向开展系统设计与实验探索。本文的研究目标不仅是生成符合格律规范与艺术美感的古诗文本,更希望通过图文协同输入、结构化提示词与反馈机制,提升生成内容与用户预期之间的匹配度,同时增强系统输出的可解释性与可控性,为用户提供更为丰富的文化体验。具体而言,本文的研究工作主要体现在以下几个方面:

在图像输入的分析方面,本文引入图像与文本的双模态输入机制,采用在中文语境下训练的跨模态大模型DeepSeek-VL2,对图像内容进行语义解析,提取物体特征、情感氛围与意境信息,生成具有文化联想价值的中文描述文本,解决了CLIP和MiniGPT-4在文化符号识别和情感表达方面的不足。该模块能够从图像中抽取对古诗创作具有指导意义的元素,为后续生成过程提供丰富、诗意的背景输入,缓解用户在文字输入中对意象表达的不确定性,有效增强了系统对场景感知与画面意境的表达能力。

在诗歌生成与表达控制方面,本文调用首个基于强化学习训练的大语言模型DeepSeek-R1进行古诗生成,相比于ERNIE-4.0等模型,其在意象运用方面展现出显著的优越性。通过参考提示工程中的CRISPE等结构化框架,设计包含背景信息、角色设定、输出风格等多个要素的复合型提示词,控制模型在生成过程中遵循既定的体裁(如七言排律)。同时,系统支持生成白话文翻译、典故注释与艺术赏析文本,帮助用户从语言形式、历史语境与情感层面理解生成内容。这种结构化的输出不仅增强了古诗生成的文化表达深度,也提升了用户对系统的信任与接受程度。

在质量评估机制方面,本文针对传统生成任务中评价方式不透明、缺乏反馈的问题,构建了一个五维度的古诗评分体系,涵盖格律规范、意象意境、主题思想、语言锤炼与创新性五个方面。每个维度下设多档评分标准,并提供示例作为参考,结合结构化Few-shot提示设计,引导大模型进行量化评分与文字评语生成。同时,配合BLEU、ROUGE、Distinct与Similarity等自动指标,从语言多样性、语义一致性等角度补充质量分析。该机制不仅能够清晰识别生成文本的强项与薄弱点,也为后续的迭代优化提供方向指引。

在优化迭代功能方面,系统在每轮古诗生成后,结合评分结果与优化建议,支持用户启动下一轮自动化修改流程。系统在保留原始图文输入意图的基础上,引导模型围绕评分中的薄弱维度进行针对性提升。该优化模块不仅强化了系统的“生成—分析—反馈—优化”闭环,也体现出大模型在语言润色与语义重构上的实用价值,提升了用户参与度与内容质量稳定性。

最后,本文设计进行了一系列实验,对系统的功能进行了全面检验。通过白话文古诗生成实验来对比DeepSeek-R1与ERNIE-4.0两个模型生成古诗的能力,却因为作为参考语料的《唐诗三百首》同样也属于两个大模型的训练语料,实验结果并不具备可信度。之后,利用第六届“诗词中国”传统诗词创作大赛的获奖作品,结合《唐诗三百首》中的律诗作品以及一些质量明显较差的打油诗,构建一个质量分层的数据集,证明了系统评分功能的有效性和优越性,还同时测试讨论了BLEU等自动度量指标的有效性。最后,通过三种不同的模态组合来测试文图模态的影响,发现图像模态的加入并没有显著提升生成古诗的质量,但图像模态仍然能够帮助用户更好地表达创作意图,提升系统的可用性与交互体验。

综上所述,本文完成了一个较为完整的基于大模型的跨模态古诗生成系统设计与实现,在功能实现和检验实验表现上取得了阶段性成果,探索了大模型在中华传统文化的再创造方面的巨大潜力,但仍存在一些不足与改进空间。首先,本系统并未实现在向量层面上的文图融合,而是通过文本描述的方式进行模态间的转换,可能导致信息损失或语义偏差,而文图模态输入对生成过程的作用仍有待进一步验证,在古诗形式质量上遇到瓶颈后,如何论证文图模态的必要性和优越性是一个问题。此外,评分机制的准确性十分依赖于大模型自身的质量,尽管本文精心设计的评分体系表现较好,但仍然存在一定的主观性与不确定性,尤其在对古诗意象的理解与文化内涵的把握上。最后,模型DeepSeek-R1的推理耗时较长,每次调用都需要消耗近一分钟的时间,可能影响用户体验。


